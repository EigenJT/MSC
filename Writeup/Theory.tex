\documentclass[12pt,a4paper,margin=1in]{report}
\usepackage[utf8]{inputenc}
\usepackage{amsmath}
\usepackage{amsfonts}
\usepackage{amssymb}
\usepackage{graphicx}
\usepackage{braket}
\usepackage{bm}
\usepackage[toc,page]{appendix}
\pagenumbering{arabic}
\linespread{1.3}
\begin{document}

\chapter{Theory of Laser Spectroscopy}
In this chapter, the theoretical background necessary to the understanding and simulation of a hyperfine spectrum is presented. In $\S$ \ref{AHF}, the features of a hyperfine spectrum are liked to physical properties a nucleus. Next, $\S$ \ref{ALI} outlines the way in which lasers interact with atoms
Note: Throughout this section, a variable written in a bold typeface is vector valued, while its non-bold counterpart is its magnitude. 
\section{Anatomy of A Hyperfine Spectrum}
\label{AHF}
\begin{figure}[h]
\includegraphics[width=\textwidth]{Graphics/ga69.png}
\label{ga69}
\end{figure}
How can the properties of a hyperfine spectrum, such as that of $^{69}\mathrm{Ga}$ shown in Fig. \ref{ga69}, be translated into measurements of the physical properties of the nucleus? The hyperfine spectrum is, after all, the result of probing the electronic structure of the atom. The answer, of course, is that the electrons interact with the nucleus through several mechanisms, each of which will be described in this section. To begin, however, consider the following system: An electron transitions from a ground state $\ket{g}$ to an excited state $\ket{e}$. More precisely $\ket{g}$ and $\ket{e}$ are defined as 
\begin{align}
\ket{g} =& \ket{n_g,J_g,J_g^z}\\
\ket{e} =& \ket{n_e,J_e,J_e^z}
\end{align}
where $n_{g,e}$ are principal quantum numbers, $\bf{J_{g,e}}$ the orbital angular momenta, and $J_{g,e}^z$ the projections of the orbital angular momentum on an axis of quantization $z$. Next, if nucleus of the atom in which this transition is occurring has angular momentum $\bf{I}$, then a quantity, $\bf{F}$, can be defined as 

\begin{equation}
\bf{F_{g,e}} = \bf{I} + \bf{J_{g,e}}
\end{equation}

$\bf{F}$ describes the total angular momentum state of the atom, so $\ket{g}$ and $\ket{e}$ can be rewritten as

\begin{align}
\ket{g} =& \ket{n_g,\mathrm{F_g}}\\
\ket{e} =& \ket{n_e,\mathrm{F_e}}
\end{align}

For a fixed $\bf{I}$, $\bf{F}$ can range from $(-\mathrm{J}+I)$ to $(\mathrm{J}+I)$.
\subsection{Peak Energies}
The energy of the electrons depends on the various electron-nucleus interactions present in the atom. If for the purposes of this work, only three interactions produce measurable changes in the energies of the electron levels. These are the isotope, magnetic dipole and electric quadrupole shifts. There are higher order interactions (magnetic octopole, electric sixteen-o-pole), however their effects are far below the resolution of the experimental set-up employed at TRIUMF. For example, the three interactions listed above lead to energy shifts on the order of an eV. Higher order interactions are (CITATION AND ANSWER NEEDED) 

\subsubsection*{Isotope Shift}
The isotope shift is measured with respect to a reference isotope. As neutrons are added or removed from a nucleus, the charge distribution, as well as the mass, of the nucleus changes. This leads to three different effects on the energies of the electrons. 

The change in the mass of the nucleus leads to what is known as the Mass shift, $\Delta E_M$. The mass shift between two isotopes with mass numbers A and A' is given by
\begin{equation}
\Delta E_M = \frac{m_{\mathrm{A}}-m_{\mathrm{A'}}}{2 m_{\mathrm{A}} m_{\mathrm{A'}}} \left(\sum_i\mathrm{\textbf{p}}_i +2 \sum_{i>j}\mathrm{\textbf{p}}_i \cdot \mathrm{\textbf{p}}_j \right)
\end{equation}
where $m_{\mathrm{A}}$ and $m_{\mathrm{A'}}$ are the isotope masses, and the $\textbf{p}_i$ are the electron momenta. The first term is due to the nucleus recoiling with the electrons when a photon is absorbed, while the second term deals with the electron-electron interactions.  

The change in the charge distribution of the nucleus produces the Field shift. While the typical nucleus is far smaller the wavefunction of a typical orbital electron, the effect is still important. The energy of a nucleus in the charge density produced by the electrons at the origin, $E_F$, is given by

\begin{equation}
E_F = \frac{Ze^2}{6 \epsilon_0}|\psi(0)|^2 \left\langle r_{ch}^2\right\rangle
\end{equation}
where $\epsilon_0$ is the permitivity of free space, $Z$ is the proton number, $e$ is the fundamental charge and $ \left\langle r_{ch}^2\right\rangle$ is the mean-square charge radius of the nucleus, defined as
\begin{equation}
 \left\langle r_{ch}^2\right\rangle = \frac{\int_0^{\infty}\rho(\mathbf{r})r^2dV}{\int_0^{\infty}\rho(\mathbf{r})dV}
\end{equation}
The field shift between two isotopes is then given by
\begin{equation}
\Delta E_F =  \frac{Ze^2}{6 \epsilon_0}\Delta|\psi(0)|^2 \Delta\left\langle r_{ch}^2\right\rangle
\end{equation}

In total, then, the isotope shift $\Delta E_{\mathrm{A,A}'}$ is given by
\begin{equation}
 \Delta E_{\mathrm{A,A'}} = \Delta E_M + \Delta E_F
\end{equation}
\subsubsection*{Magnetic Dipole Shift}
A nucleus with a non-zero nuclear spin $\bf{I}$  will have a magnetic dipole moment, given by

\begin{equation}
\boldsymbol{\mu}_{\mathrm{\bf{I}}} = g_{\mathrm{I}}\mu_{\mathrm{N}}\mathrm{\bf{I}}
\end{equation}

where $g_{\mathrm{I}}$ is the g-factor and $\mu_{\mathrm{N}}$ is the nuclear magneton. (REFERENCE NEEDED) The interaction of $\mu_{\mathrm{I}}$ with the magnetic field produced by the electrons, $\mathrm{\bf{B_e}}$, creates a shift in the energy of the orbiting electrons. Provided the electrons occupy an angular momentum state $\mathrm{\bf{J}} \neq 0$, the hamiltonian for this interaction is given by

\begin{equation}
\mathcal{H} = -\boldsymbol{\mu}_{\mathrm{\bf{I}}} \cdot \mathrm{\bf{B_e}}
\end{equation}

This interaction leads to a shift, $\Delta E_{\mu_I}$, in the energy of the atomic states by

\begin{equation}
\Delta E_{\mu_I} = \frac{AK}{2}
\end{equation}

where $K = \mathrm{F(F+1) - I(I+1) - J(J+1)}$ and 

\begin{equation}
A = \frac{\mu_{\mathrm{I}}\mathrm{B_e}}{\mathrm{IJ}}
\end{equation}

\subsubsection*{Electric Quadrupole Shift}
The electric quadrupole moment is used to describe the distribution of charge in a nucleus. For a nucleus composed of $n$ protons and $\mathbf{I}\geq1$, the electric quadrupole moment, Q, is given by

\begin{equation}
\mathrm{Q} = \sum_i^n (3z_i^2-r_i^2)
\end{equation}
where $r_i^2 = x_i^2+y_i^2+z_i^2$. If Q $ < 0$, then the nucleus is stretched in the $x-y$ plane. If Q $ > 0$, then the nucleus is stretched along the $z-$axis. It is important to note that these deformations a symmetric with respect to an axis of symmetry, the $z-$axis in this case. Q $=0$ indicates that the nucleus is spherical. 

In reality, direct measurement of Q is not feasible, as the nucleus is rotating. Instead, the spectroscopic quadrupole, Q$_s$, is measured. Q$_s$ is defined as the projection of Q onto the axis of quantization of the nucleus, and is given by
\begin{equation}
\mathrm{Q}_s = \frac{\mathrm{I}(2\mathrm{I}-1)}{(\mathrm{I}+1)(2\mathrm{I}+3)}\mathrm{Q}
\end{equation}
The use of $Q_s$ as a measure of $Q$ is valid under the assumption that the nuclear deformation is axially symmetric. Additionally, it is assumed that the axis of symmetry has a well defined direction with respect to \textbf{I}.

The hamiltonian for the interaction between the spectroscopic electric quadrupole moment and the electric field produced by the electrons at the nucleus, $E_N$, is given by

\begin{equation}
\mathcal{H} = - \frac{1}{6}e\mathrm{Q}_s\nabla{E_N}
\end{equation}
where
\begin{equation}
\nabla{E_N} = \frac{\partial^2V}{\partial x_i\partial x_j}, \{x_j,x_k\} \in \{x,y,z\} \otimes \{x,y,z\}
\end{equation}
$e$ is the fundamental charge and $V$ is the electric potential. Recalling that the nuclear deformation is symmetric about the axis of quantization, the shift in energy is then given by

\begin{equation}
\Delta E_{\mathrm{Q_s}} = \frac{B}{4}\left[\frac{\frac{3}{2}K(K+1)-2\mathrm{I}(\mathrm{I}+1)\mathrm{J}(\mathrm{J}+1)}{\mathrm{I}(2\mathrm{I}-1)\mathrm{J}(2\mathrm{J}-1)}\right]
\label{B-eq}
\end{equation}
where $B$ is a hyperfine coefficient and is given by
\begin{equation}
B = e\mathrm{Q_s}\left\langle\frac{\partial^2V}{\partial z^2} \right\rangle
\end{equation}
Eq. \ref{B-eq} has singularities at \textbf{I} = $\frac{1}{2}$ and \textbf{J}=$\frac{1}{2}$. This captures the fact that in either case there is no coupling between the nucleus and the electrons, as an orbital angular momentum state of $\frac{1}{2}$ is isotropically distributed in space .

\subsubsection*{The Hyperfine Equation}
The resonant energy of a transition between $\ket{g}=\ket{n_g,\mathrm{\textbf{F}_g=\textbf{I} + \textbf{J}_g}}$ and $\ket{e}=\ket{n_e,\mathrm{\textbf{F}_e=\textbf{I} + \textbf{J}_e}}$ is then given by
\begin{equation}
E_{hfs} = E_{fs} +  \Delta E_{\mathrm{A,A'}}+\Delta E_{\mu_I}\Bigr|_{\mathrm{F_g},I,J_g}^{\mathrm{F_e},I,J_e}+\Delta E_{\mathrm{Q_s}}\Bigr|_{\mathrm{F_g},I,J_g}^{\mathrm{F_e},I,J_e}
\end{equation}

When a hyperfine spectrum is fit, the fit parameters which govern the locations of the peaks are the hyperfine parameters $A$ and $B$ for both $\ket{e}$ and $\ket{g}$, as well as what is known as the centroid, $\nu_0$. The centroid combines the fine structure energy and the isotope shift into one quantity. It is important to note that while the values of hyperfine coefficients can be directly linked to the physical properties of the nucleus, only the change in the centroid with respect to a reference isotope can be used to measure the change in the mean-square charge radius.

\subsection{Peak intensities} 
The other notable characteristic of a hyperfine spectrum, the ratios of the peak intensities can give information as to the spin of the nucleus if the angular momentum states of the electron orbitals are known. 

\section{Spontaneous Emission in \\ Multi-Level Atoms}
\label{ALI}
A hyperfine spectrum is constructed through the measurement of the photons emitted as electrons in excited states de-excite to lower energy states. In order to simulate a hyperfine spectrum, then, it is necessary to understand the mechanisms through which electrons transition between energy levels in an atom.

Consider a two-level atom in an excited state. After a time $t$, the atom transitions to the ground state. The wavefunction of the system is given by
\begin{equation}
\ket{\psi} = c_{e,0}e^{-i\omega_et}\ket{e,0} + \sum_Sc_{g,1}e^{-i(\omega_g+\omega)t}\ket{g,1_S}
\end{equation}
where $S = (\bm{k},\bm{\varepsilon})$ gives the wavevector $\bm{k}$ and polarization $\bm{\varepsilon}$ of an emitted photon, $\omega = kc$, $\omega_e$ and $\omega_g$ are the energies of the excited and ground states, respectively (REFERENCE NEEDED). The time evolution of the two states is described by
\begin{align}
i\frac{\mathrm{d}c_{e,0}(t)}{\mathrm{d}t} =& \sum_Sc_{g,1_S}(t)\Omega_S e^{-i(\omega-\omega_a)t}\\
i\frac{\mathrm{d}c_{g,1_S}(t)}{\mathrm{d}t} =& \ c_{e,0}(t)\Omega_S^*e^{i(\omega-\omega_a)t}
\end{align}
where the coupling between each state is given by the Rabi frequency \\$\Omega_s = -\bm{\mu}\cdot\bm{E}_{\omega}/\hbar$. The electric dipole moment, $\bm{\mu}$ between two states is given by
\begin{equation}
\bm{\mu} = e\bra{e}\bm{r}\ket{g} 
\end{equation}
and the electric field per mode is given by
\begin{equation}
\bm{E}_{\omega} = \sqrt{\frac{\hbar\omega}{2\epsilon_0V}}\bm{\varepsilon}
\end{equation}
$V$ is the volume over which the field is quantized and will eventually drop out. Integration of Eq.1.24 and substitution of the result into Eq.1.23, followed by further integration  





For an atom in a radiation field of amplitude $E_0$, the Rabi frequency, the frequency at which an electron transitions between an excited state $\ket{e}$ and ground state $\ket{g}$ is defined as (REFERENCE NEEDED)
\begin{equation}
\Omega \equiv \frac{eE_0}{\hbar}\bra{e}\bm{\varepsilon}\cdot\bm{r}\ket{g}
\end{equation}
where $\bm{\varepsilon}$ is the polarization of the light. In general, calculating the dipole moment  $\mu = \bra{e} \bm{\varepsilon} \cdot \bm{r} \ket{g}$ is not a simple task. Immediately including the effects of the hyperfine interactions would further complicate matters. As such, it is useful to first describe the basic process through which the dipole moment $\mu$ is calculated. Once this is understood, including the hypefine interaction will simply be a extension of an already understood computation.
In the simplest case (read hydrogenic wavefunctions), the Wigner-Eckart theorem states that the dipole moment can be split into two separate quantities in the following manner
\begin{equation}
\mu_{eg} = e \mathcal{R}_{n_e J_e, n_g J_g}\mathcal{A}_{J_eJ_e^z, J_gJ_g^z}
\end{equation}
where $\mathcal{R}_{n_e J_e, n_g J_g}$ and $\mathcal{A}_{J_eJ_e^z, J_gJ_g^z}$ are known as the radial and angular parts of the dipole moment, respectively. While both quantities are presented in more depth for the sake of completion (not the word I want to use...), it will become apparent that the hyperfine inter

\subsection{Radial Part}
In most cases in laser spectroscopy, the radial part of the dipole moment only acts as an overall multiplicative factor for the strength of the coupling between the excited and ground states. This is due to the fact that all available ground states typically share the same radial wavefunction, likewise in the case of the excited states. The radial part is given by
\begin{equation}
\mathcal{R}_{n_e J_e, n_g J_g} = \bra{R_{n_e J_e}}\bm{r}\ket{R_{n_g J_g}} = \int_0^{\infty}r^2R_{n_e J_e}(r)rR_{n_g J_g}(r)dr
\end{equation}
where the $R_{nJ}$ are the hydrogenic radial wavefunctions of their respective states
\begin{equation}
R_{nJ}(r) = N_{nJ}\rho^J\exp(-\rho/2)L_{n-J-1}^{2J+1}(\rho)
\end{equation}
where $N_{nJ}$ is a normalization constant and $L_{n-J-1}^{2J+1}(\rho)$ are the Laguerre polynomials evaluated at $\rho = 2r/na_0$, where $a_0$ is the Bohr radius. Table \ref{Rad_calc} shows the results results of the calculation for $n \leq 5$. The  


\begin{appendices}
\chapter{Look-Up Tables for Radial Part of Dipole Moment}


\begin{table}[h]
\centering
\begin{tabular}{ccccc}\hline \hline
nJ & 2(J+1) & 3(J+1) & 4(J+1) & 5(J+1) \\ \hline
\end{tabular}
\caption{hello}
\label{Rad_calc}
\end{table}




\end{appendices}

\end{document}