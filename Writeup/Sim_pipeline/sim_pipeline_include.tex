In this chapter the simulation of a hyperfine spectrum measured at TRIUMF is described in detail. $\S$\ref{pip} includes little more than a graphical representation of the algorithm followed when the atoms are interacting with the laser. Each subsequent section is dedicated to an individual step in the algorithm, detailing the logic and computations required at that step. The final section of the chapter shows how a hyperfine spectrum is built through the repeated application of the interaction loop.

\begin{figure}[h]

\begin{tikzpicture}[node distance = 2.5cm]
\tiny

\node (props) [prelim] {Select velocity and ground state};
\node (exprobs) [prelim, right of=props] {Compute exitation rates};
\node (Adv_exc) [block, right of=exprobs] {Advance by eaxcitation time};
\node (IR) [decision, above of=Adv_exc] {In IR?};
\node (Exc) [block, right of=IR] {Excite to new state};
\node (Adv_life) [block, right of=Exc] {Advance by mean lifetime};
\node (IR_2) [decision, below of=Adv_life] {In IR?};
\node (Next) [end, right of=IR_2] {Next atom};
\node (deex) [block, below of=IR_2] {Deexcite and select new ground state};
\node (LCR) [decision, left of=deex] {In LCR?};
\node (PHO) [block, left of=LCR] {Record photon and increase counter};
\node[coordinate] (bl1) [above of = IR,yshift=-15mm]{};
\node[coordinate] (bl2) [above of=Next]{};
\node[coordinate] (bl3) [above of=bl2,yshift=-15mm]{};
\node[coordinate] (bl4) [above of = LCR,yshift = -15mm]{};
\node[coordinate] (bl5) [left of =PHO,yshift = 5mm]{};

\draw [arrow] (props)--(exprobs);
\draw [arrow] (exprobs)--(Adv_exc);
\draw [arrow] (Adv_exc)--(IR);
\draw [arrow] (IR) -- node[anchor=south] {Yes} (Exc);
\draw (IR) -- node[anchor=north east,yshift = 1mm] {No} (bl1);
\draw (bl3) -- (bl1);
\draw [arrow] (bl3)--(Next);
\draw [arrow] (Exc)--(Adv_life);
\draw [arrow] (Adv_life) -- (IR_2);
\draw [arrow] (IR_2) -- node[anchor = south] {No} (Next);
\draw [arrow] (IR_2) -- node[anchor = east] {Yes} (deex);
\draw [arrow] (deex) -- (LCR);
\draw [arrow] (LCR) -- node[anchor = south] {Yes} (PHO);
\draw (LCR)-- node[anchor = west] {No} (bl4);
\draw [arrow] (PHO) -| (exprobs);
\draw [arrow] (bl4)-|(exprobs);



\end{tikzpicture}
\label{pipeline}
\caption{Flow chart detailing the steps undertaken by the algorithm to follow the atom as it moves through the IR. The green ellipses and blue rectangles compose the interaction loop of the simulation, where the atom interacts with the laser and goes through excitation/decay cycles. Finally, the yellow rectangle is the end point of the loop, and occurs when then atom has moved beyond the IR. }
\end{figure}

\label{pip}


\section{Interaction Loop}


Fig. \ref{pipeline} shows each step taken an atom as it passes through the simulation. The red rounded rectangles represent preliminary steps in the simulation, where initial properties are imparted to the atom. The first preliminary step selects a velocity and ground state for the atom. The following step is the calculation of the likelihood of excitation for each allowed transition. This prepares the atom for what can be considered the main loop of the algorithm. Here, the atom and laser interact as the atom traverses the IR. 

The interaction loop is composed of green ellipses and blue rectangles. Two of the three green ellipses represent regular checks to ensure that the atom is still in the IR. If at any point the atom is found to have moved beyond the IR, then the algorithm moves on to the next atom. The third green ellipse is dedicated to checking if a photon released by the atom would be measured by the PMT. The evolution of the state and position of the atom is described by the blue rectangles. 

\section{Preliminary Steps}
The first two steps select the initial properties given to the atom, determining how likely the atom is to interact with the laser. The two most important properties are the velocity and the ground state of the atom. Both are chosen through the sampling of their respective probability distributions.

\subsection{Velocity Selection}

The velocity of the atom $v_a$ can be decomposed in to two elements: the mean velocity of the beam, $v_{m}$, and the thermal velocity $v_{T}$. 
\begin{equation}
v_a = v_m + v_T
\end{equation}
The mean velocity is determined by the energy of the beam and the mass of the atom, i.e.
\begin{equation}
v_{m} = \sqrt{\frac{2E_b}{m_A}}
\end{equation}
where $E_b$ is the energy of the beam and $m_A$ is the mass of an atom with mass number $A$.

The thermal velocity is selected by sampling the Maxwell-Boltzmann distribution, described in (INSERT REFERENCE TO THEORY SECTION). This is done using the Box-Muller transform, which samples a uniform distribution twice and converts the results into a sample of a normal distribution. A sample $v_T$, of a Maxwell-Boltzmann distribution for atoms of mass $m_A$ and temperature $T$ is given by
\begin{equation}
v_T = \sqrt{\frac{KT}{2m_A}}\sqrt{-2\log x_1}\cos (2\pi x_2)
\end{equation}
where $x_1,x_2 \in [0,1]$ are two randomly generated numbers taken from a uniform distribution.

\subsection{Ground State Selection}
For an atom with a coupled angular momentum state $F_g$ as the ground state, the electron can occupy any of the $2F_g+1$ projections on the axis of quantization. A particular projection, say $F_i \in [-F_g,-F_g+1,...,F_g-1,F_g]$, has a probability of being occupied that is proportional to $2F_i+1$. Taking the probability space occupied by each projection in to account, the probability that an electron occupies the projection $F_i$, $P(F_i)$, is given by
\begin{equation}
P(F_i) = \frac{2F_i+1}{\sum_j 2F_j+1}
\label{gs_prob}
\end{equation}
Knowing this, the ground state can be chosen through the generation of a random number $x \in [0,1]$ from a uniform distribution. Each projection is given a range of numbers from 0 to 1, proportional to Eq. \ref{gs_prob}. If $x$ falls in the assigned range, then that ground state is chosen. 

\subsection{Computation of Excitation Rates}
With the ground state and the velocity of the atom now selected, the probability that a transition will occur can be calculated. For each allowed transition between the chosen ground state $F_g$ and an excited state $F_{e,i}$, the transition rate $\gamma_i$ is given by Eq. (REFERENCE THEORY). Note that the laser frequency $f_l$ must be shifted according to the relavistic doppler effect which depends on the velocity of the atom. The frequency observed by the atom, $f_o$, is given by \cite{cmec}
\begin{equation}
f_o = f_l \sqrt{\frac{1+\beta}{1-\beta}}
\end{equation}
where $\beta = v_a/c$.

\section{Interaction Loop}
The interaction loop of the algorithm is the place where the atoms undergo several instances of excitation and decay. Also included are regular checks to see in the atom would still be present in the in interaction region, as well as whether or not a released photon would be measured by the light collection system present in the experiment.

\subsection{Excitation Time}
Once all the excitation rates have been computed, the atom can now be advanced by the expected time it takes for a transition to occur. This time is called the excitation time, $t_e$, and is given by
\begin{equation}
t_e = \sum_i \frac{1}{\gamma_i}
\end{equation}
After time $t_e$, the atom is advanced a distance $d_e = t_ev_a$. If it is still in the interaction region, then an excited state $F_e$ is chosen by sampling a uniform distribution where each transition is given a region proportional to $1/\gamma_i$, in similar style to the method used to select the ground state. If the atom is no longer in the interaction region after $t_e$, then it is discarded and the algorithm moves on to the next atom.

\subsection{Advancing by the Mean Lifetime, Decay and Selection of New Ground State}
Once the excited stated of the atom has been selected, then the atom is advanced a distance $d_l = t_lv_a$ where $t_l$ is the mean lifetime of the excited state. The mean lifetimes of each excited state are computed ( according to (REFERENCE ENTIRE EMISSION SECTION)) once at the beginning of the simulation, then stored for later use. If the atom is still in the interaction region after having moved $d_l$, then it decays into one of the allowed ground states. If not, then the simulation moves on to the next atom. As with the selection of the excited state, a uniform distribution is sampled, and each ground state $F_{g,i}$ is given a range of values proportional to the inverse of the lifetime of the transition from the excited state, $F_{e}$, to $F_{g,i}$. 

Once a ground state is selected, a photon with energy given by (referece theory again) is "released". If the atom is in the light collection region, then this photon is recorded for later analysis. Additionally, a counter that keeps track of the number of photons collected at each beam energy is increased by one. If the atom is not in the LCR, then nothing is recorded, and the excitation-decay process begins again with the computation of excitation rates.

\section{Simulation of Complete Run}
In order to simulate a complete experimental run, a chosen number of atoms, say N, are passed in sequence through the above loop. This is done in turn for each beam energy in a list of energies that are selected such that the doppler shifted laser energies range from the lowest energy to the highest energy transition, with some leeway on either side. This range can be called $E_r$. The graph of photon counts per beam energy is in fact the hyperfine spectrum.

\vspace{10mm}
\begin{algorithm}[H]
\SetAlgoLined
\KwResult{Simulation of complete hyperfine spectrum}
 Input all preliminary parameters\;
 \For{Beam energy in $E_r$}{
  \For{Each atom}{
   Run \textbf{interaction loop}\;
   Record photon count\;
  }
  Sum photon count into counts per beam energy\;
 }
 Plot photon count at each beam energy\;
 \caption{Pseudo-code for the simulation of a complete hyperfine spectrum.}
\end{algorithm}

\vspace{10mm}
The preliminary parameters mentioned in the above pseudo-code refer to all the quantities required to perform a complete simulation of a hyperfine spectrum measured at TRIUMF. A list of these parameters is provided below, for completeness.
\begin{list}{•}{List of Preliminary Parameters and Their Symbols}
\item Isotope mass: $m_A$
\item Ground J-state: $\bf{J}_g$
\item Excited J-state: $\bf{J}_e$
\item Nuclear spin: $\bf{I}$
\item Principal quantum number of ground state: $n_g$
\item Principal quantum number of excited state: $n_e$
\item Magnetic dipole hyperfine coefficient of ground state: $A_g$
\item Magnetic dipole hyperfine coefficient of excited state: $A_e$
\item Electric quadrupole hyperfine coefficient of ground state: $B_g$
\item Electric quadrupole hyperfine coefficient of excited state: $B_e$
\item Beam temperature: $T_b$
\item Laser frequency: $f_l$
\item Laser intensity: $I_l$
\item Fine structure transition energy: $E_{fs}$
\item Number of atoms to simulate per beam energy: $N_a$
\item Distance between CEC and LCR: $d$
\item Length of LCR: $d_{LCR}$
\end{list}