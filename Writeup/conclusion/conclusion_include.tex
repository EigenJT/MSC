\noindent In this work, a brief introduction to collinear laser spectroscopy at TRIUMF was presented, followed by the theoretical background necessary to understand the hyperfine structure of an atom. Then, a treatment of optical pumping as a modification to the Racah intensities was described and then used to simulate the effects of optical pumping on the measured hyperfine spectra of both Gallium-69 and Rubidium-87. The effects of the beam temperature, the experimental geometry and the power of the exciting laser were explored on a simulated Gallium-69 spectrum. Increasing the temperature caused the peak width to increase, while increasing the distance between the CEC and the LCR affected the relative heights of the hyperfine peaks. Increasing the laser power caused the relative peak heights to change. Continuously increasing the laser power led to the complete domination of a single transition, with the other transitions being \emph{pumped out}. When compared to measured Gallium-69 spectra, there was a discrepancy between the assumed beam temperature (~300 K) and the temperature that produced most accurate simulation (~910 K), possibly caused by an inaccurate estimation of the cooling effects of the RFQ, or a broadening effect caused by the accelerating voltages. When compared Rubidium-87 spectra measured using different laser powers, the simulation consistently underestimated the height of the lowest peak. The discrepancy between the measured and simulated Rubidium-87 spectra seemed to increase with laser power. Given that the simulation was accurate in one case and inaccurate in the other, there is likely some merit in comparing the simulation to different isotopes in future experiments. 